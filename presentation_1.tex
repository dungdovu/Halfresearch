%%%%%%%%%%%%%%%%%%%%%%%%%%%%%%%%%%%%%%%%%
% Beamer Presentation
% LaTeX Template
% Version 1.0 (10/11/12)
%
% This template has been downloaded from:
% http://www.LaTeXTemplates.com
%
% License:
% CC BY-NC-SA 3.0 (http://creativecommons.org/licenses/by-nc-sa/3.0/)
%
%%%%%%%%%%%%%%%%%%%%%%%%%%%%%%%%%%%%%%%%%

%----------------------------------------------------------------------------------------
%	PACKAGES AND THEMES
%----------------------------------------------------------------------------------------

\documentclass{beamer}

\mode<presentation> {

% The Beamer class comes with a number of default slide themes
% which change the colors and layouts of slides. Below this is a list
% of all the themes, uncomment each in turn to see what they look like.

%\usetheme{default}
%\usetheme{AnnArbor}
%\usetheme{Antibes}
%\usetheme{Bergen}
%\usetheme{Berkeley}
%\usetheme{Berlin}
%\usetheme{Boadilla}
%\usetheme{CambridgeUS}
%\usetheme{Copenhagen}
%\usetheme{Darmstadt}
%\usetheme{Dresden}
%\usetheme{Frankfurt}
%\usetheme{Goettingen}
%\usetheme{Hannover}
%\usetheme{Ilmenau}
%\usetheme{JuanLesPins}
%\usetheme{Luebeck}
\usetheme{Madrid}
%\usetheme{Malmoe}
%\usetheme{Marburg}
%\usetheme{Montpellier}
%\usetheme{PaloAlto}
%\usetheme{Pittsburgh}
%\usetheme{Rochester}
%\usetheme{Singapore}
%\usetheme{Szeged}
%\usetheme{Warsaw}

% As well as themes, the Beamer class has a number of color themes
% for any slide theme. Uncomment each of these in turn to see how it
% changes the colors of your current slide theme.

%\usecolortheme{albatross}
%\usecolortheme{beaver}
%\usecolortheme{beetle}
%\usecolortheme{crane}
%\usecolortheme{dolphin}
%\usecolortheme{dove}
%\usecolortheme{fly}
%\usecolortheme{lily}
%\usecolortheme{orchid}
%\usecolortheme{rose}
%\usecolortheme{seagull}
%\usecolortheme{seahorse}
%\usecolortheme{whale}
%\usecolortheme{wolverine}

%\setbeamertemplate{footline} % To remove the footer line in all slides uncomment this line
%\setbeamertemplate{footline}[page number] % To replace the footer line in all slides with a simple slide count uncomment this line

%\setbeamertemplate{navigation symbols}{} % To remove the navigation symbols from the bottom of all slides uncomment this line
}

\usepackage{graphicx} % Allows including images
\usepackage{booktabs} % Allows the use of \toprule, \midrule and \bottomrule in tables

%----------------------------------------------------------------------------------------
%	TITLE PAGE
%----------------------------------------------------------------------------------------

\title[Department of Electrical Engineering]{Group Presentation} % The short title appears at the bottom of every slide, the full title is only on the title page

\author{Dung Do Vu} % Your name
\institute[ETSMTL] % Your institution as it will appear on the bottom of every slide, may be shorthand to save space
{
Department of Electrical Engineering \\
Ecole de Technologies Superieure (ETS),  University of Quebec, Montreal, Canada\\ % Your institution for the title page
\medskip
\textit{do-dung.vu.1@ens.etsmtl.ca} % Your email address
}
\date{\today} % Date, can be changed to a custom date

\begin{document}

\begin{frame}
\titlepage % Print the title page as the first slide
\end{frame}

\begin{frame}
\frametitle{Overview} % Table of contents slide, comment this block out to remove it
\tableofcontents % Throughout your presentation, if you choose to use \section{} and \subsection{} commands, these will automatically be printed on this slide as an overview of your presentation
\end{frame}

%----------------------------------------------------------------------------------------
%	PRESENTATION SLIDES
%----------------------------------------------------------------------------------------

%------------------------------------------------
\section{Background} % Sections can be created in order to organize your presentation into discrete blocks, all sections and subsections are automatically printed in the table of contents as an overview of the talk
%------------------------------------------------

\subsection{Academic and education} % A subsection can be created just before a set of slides with a common theme to further break down your presentation into chunks



\begin{frame}
\frametitle{Academic Background}
\begin{itemize}
\item \textbf{2004 - 2009}: B. Sc in Electronic and Telecommunication Engineering from Hanoi University of Science and Technology, Vietnam 
\begin{itemize}
	\item \textcolor{blue}{Study}: Design and develop the vehicle tracking system using GPS and GSM technology 
\end{itemize}
\item \textbf{2012 - 2014}: M. Sc in Electronic and Telecommunication Engineering from Hanoi University of Science and Technology, Vietnam 
\begin{itemize}
	\item \textcolor{blue}{Study}: Design the smart antenna application for the Intelligent Traffic System (ITS)
\end{itemize}


\end{itemize}
\end{frame}

%------------------------------------------------

\begin{frame}
\frametitle{Academic Background}
\begin{itemize}
	\item \textbf{2004 - 2009}: B. Sc in Electronic and Telecommunication Engineering from Hanoi University of Science and Technology, Vietnam 
	\begin{itemize}
		\item \textcolor{blue}{Study}: Design and develop the vehicle tracking system using GPS and GSM technology 
	\end{itemize}
	\item \textbf{2012 - 2014}: M. Sc in Electronic and Telecommunication Engineering from Hanoi University of Science and Technology, Vietnam 
	\begin{itemize}
		\item \textcolor{blue}{Study}: Design the smart antenna application for the Intelligent Traffic System (ITS)
	\end{itemize}
\item\textbf{ 6/2017 - now}: Ph. D (in progress) in Electrical Engineering at ETS (since June 2017) 
\begin{itemize}
	\item \textcolor{blue}{Study}: DGA-1005
\end{itemize}
	
	
\end{itemize}
\end{frame}

%------------------------------------------------
\subsection{Technical skill}

\begin{frame}
\frametitle{Computer skills}
\begin{block}{Primary Programming}
C/C++, R, Python, Qt, bash shell, Yocto, and Matlab
\end{block}

\begin{block}{Data Analysis System}
Apache Hadoop, Mapreduce, Hive, Google Bigquery, SQL, and SPSS
\end{block}

\begin{block}{Visualization}
Tableau, Python, R, and Matlab
\end{block}

\begin{block}{Documentation}
	Latex Studio (slide, poster, figure, flowchart, and document), Microsoft Office (word, excel, powerpoint, access)
\end{block}

\begin{block}{Miscellaneous}
Windows, Linux, Git, SVN, Agile
\end{block}



\end{frame}

%------------------------------------------------

\begin{frame}
\frametitle{Project}
\textcolor{blue}{\textbf{R\&D project}}
\begin{itemize}



	\item The researcher of developing the smart antenna for Intelligence Traffic System at Hanoi University of Science and Technology
	\item The researcher of developing the GeoSocialBound framework at Dankook University
	
	\item The senior developer of researching the smart vehicle by using OpenCV at FPT Coporation
\end{itemize}
\textcolor{blue}{\textbf{Implementation project}}
\begin{itemize}

		\item The headend engineer of implementing the IPTV system for Vietnam Post Telecommunication Corporation
	
	\item The head office engineer of auditing the information security for Tien Phong Bank
	\item The project leader of developing the civil infrastructure platform based on linux kernel 4.4.51 at Renesas Coporation
\end{itemize}	

\end{frame}


%------------------------------------------------






\section{Research fields}
%------------------------------------------------


%------------------------------------------------

\subsection{Before coming to \'ETS}
%------------------------------------------------

\begin{frame}[fragile] % Need to use the fragile option when verbatim is used in the slide
\frametitle{Research fields}

\textbf{Before coming to \'ETS}

%------------------------------------------------
\begin{itemize}
	\item \textbf{Big data analytics}: Predictive analytic, data mining, text analytic, 
	\item \textbf{Machine learning}: Supervised and unsupervised learning
	\item \textbf{Online social network analysis}: Networked structures in terms of nodes including social media networks, friendship, and collaboration graphs
\end{itemize}


%------------------------------------------------



%------------------------------------------------


\end{frame}
%------------------------------------------------
\subsection{Researching at \'ETS}
\begin{frame}[fragile] % Need to use the fragile option when verbatim is used in the slide
\frametitle{Research fields}

\textbf{Researching at \'ETS} -- on progress
%------------------------------------------------
\begin{flushleft}
	Focus on developing new algorithm to analyze big data generated by IoT-based 5G networks\\
	\begin{itemize}
		\item Big data analystics based on random matrix theory in  IoT and 5G networks
			\begin{itemize}
		\item Architectural framework to support big data in 5G networks: big signaling, traffic, location, radio waveform, and heterogenous data
		\item Filtering and extracting right meta data from the whole data without discarding useful information
	\end{itemize}
	\item Big data anlystics based on Mobile network data collection
	\begin{itemize}
		\item Optimize, configure, and plan the network operations based on the stream data
		\item Autonomous fault detection and localization solution for identifying coverage holes, sleeping cells, cells in outages, and network
		\item Developing model of user and system behaviour
		
		
	\end{itemize}

	\end{itemize}
\end{flushleft}


%------------------------------------------------



%------------------------------------------------


\end{frame}
\begin{frame}
\Huge{\centerline{The End}}
\end{frame}

%----------------------------------------------------------------------------------------

\end{document} 